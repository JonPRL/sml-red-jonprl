\chapter{Nominal LCF: a language for tactics}

In a sequent calculus, left rules add hypotheses to the context; for instance,
consider the left rule for positive conjunctions:
\[
  \infer[{\otimes_L^{\IMode{x},\OMode{y},\OMode{z}}}]{
    \IMode{H,x:A\otimes B}\gg\IMode{C}
  }{
    \IMode{H,x:A\otimes B,y:A,z:B}\gg\IMode{
      \IMode{\Subst{\Pair{y}{z}}{x}{C}}
    }
  }
\]
%

From a proof refinement perspective (see~\cite{constable:1986}), such a rule is
typically manifested as an ML tactic $\mathtt{\otimes_L}[x,y,z]$ which takes
three names as parameters: the target hypothesis $\IMode{x}$, and the names to
use for the new hypotheses $\OMode{y},\OMode{z}$. However, whilst the identity
of the name $x$ is essential to the meaning of the tactic, the names supplied
for the generated hypotheses can be freshly renamed with impunity.

Indeed, in a proof term assignment for this sequent calculus, the corresponding
elimination form would \emph{bind} variables $x,y$ rather than take the, as
parameters. However, in the standard LCF tactic paradigm, it is not possible to
reproduce this structure, because the sequencing of rules is mediated by the
general purpose $\mathtt{THEN}$ tactical, which has no knowledge of names or
binding.

We will design a language for tactics called \textbf{Nominal LCF} which
supports a distinction between names bound and names taken as parameters, and
then show how it can be elaborated into standard LCF.

\section{The LCF Tactic Language}
\newcommand\MLTypeDecl[1]{\mathtt{type}\ #1}
\newcommand\PartialFun[2]{#1\rightharpoonup #2}
\newcommand\LCFJudgment{\mathit{judgment}}
\newcommand\LCFEvidence{\mathit{evidence}}
\newcommand\LCFTactic{\mathit{tactic}}
\newcommand\MLList[1]{#1\;\mathit{list}}
\newcommand\MLAtom{\mathbb{A}}
\newcommand\MLNat{\mathbb{N}}
\newcommand\MLStream[1]{{#1}^{\MLNat}}
\newcommand\LCFAtomicTactic{\mathit{atactic}}
\newcommand\StreamTake[2]{\bar{#1}(#2)}

The essence of the LCF tactic system is captured in the following (idealized)
ML signature:

\[
  \begin{array}{lcl}
    \MLTypeDecl{\LCFJudgment}\\
    \MLTypeDecl{\LCFEvidence}\\
    \MLTypeDecl{\LCFTactic} &= &\PartialFun{\LCFJudgment}{\MLList{\LCFJudgment} \otimes (\MLList{\LCFEvidence}\to\LCFEvidence)}
  \end{array}
\]

In other words, a tactic is a partial function that takes a goal to its
subgoals, and specifies how to transform the evidence of its subgoals into the
evidence for the main goal. In the case of the sequence calculus we were
considering, \verb!judgment! would be a type of sequents.  Now, in general a
tactic may need to consume names from a \emph{name store}, which is an infinite
stream of \emph{atoms} or \emph{symbols}:

\[
  \begin{array}{lcl}
    \MLTypeDecl{\MLAtom}\\
    \MLTypeDecl{\LCFAtomicTactic} &= &\MLStream{\MLAtom} \to \LCFTactic
  \end{array}
\]

\newcommand\KleeneEq[2]{\IMode{#1}\approx \IMode{#2}}
\newcommand\UMod[2]{\mathsf{umod}(#1,#2)}

\subsection{Atomic tactics and their uniform continuity}

Left sequent rules can be coded as so-called \emph{atomic tactics}, tactics which
consume a stream of names. In fact, every such tactic is \emph{uniformly
continuous} in a specific sense. For an atomic sequence
$\Member{\alpha}{\MLStream{\MLAtom}}$ and a natural number
$\Member{n}{\MLNat}$, let $\StreamTake{\alpha}{n}$ be the initial segment of
$\alpha$ of length $n$. Let $\KleeneEq{M}{N}$ be \emph{observational equivalence}:
$M$ evaluate to the same value, or they diverge.

Then, for any atomic tactic $t$ and judgment $\mathcal{J}$, we can calculate a
uniform modulus of continuity:
\begin{equation}
  \exists \OMode{n}\in\IMode{\MLNat}.
  \forall \Member{\alpha,\beta}{\MLStream{\MLAtom}}.
    \IsEq{\bar{\alpha}(n)}{\bar{\beta}(n)}\implies \KleeneEq{t(\alpha,\mathcal{J})}{t(\beta,\mathcal{J})}
  \tag{uniform continuity}
\end{equation}

This calculation can be realized computationally in our metalanguage in a
number of ways, but for our purposes it suffices to remark that it is a
consequence of Brouwer's Fan Theorem, which we hold to be evident. Let
$\Member{\UMod{t}{\mathcal{J}}}{\MLNat}$ be the uniform modulus of continuity
for a tactic $t$ at goal $\mathcal{J}$.

\section{Nominal LCF}
