\chapter{Nominal LCF: a language for tactics}

In a sequent calculus, left rules add hypotheses to the context; for instance,
consider the left rule for positive conjunctions:
\[
  \infer[{\otimes_L^{\IMode{x},\OMode{y},\OMode{z}}}]{
    \IMode{H,x:A\otimes B}\gg\IMode{C}
  }{
    \IMode{H,x:A\otimes B,y:A,z:B}\gg\IMode{
      \IMode{\Subst{\Pair{y}{z}}{x}{C}}
    }
  }
\]
%

From a proof refinement perspective (see~\cite{constable:1986}), such a rule is
typically manifested as an ML tactic $\mathtt{\otimes_L}[x,y,z]$ which takes
three names as parameters: the target hypothesis $\IMode{x}$, and the names to
use for the new hypotheses $\OMode{y},\OMode{z}$. However, whilst the identity
of the name $x$ is essential to the meaning of the tactic, the names supplied
for the generated hypotheses can be freshly renamed with impunity.

Indeed, in a proof term assignment for this sequent calculus, the corresponding
elimination form would \emph{bind} variables $x,y$ rather than take the, as
parameters. However, in the standard LCF tactic paradigm, it is not possible to
reproduce this structure, because the sequencing of rules is mediated by the
general purpose $\mathtt{THEN}$ tactical, which has no knowledge of names or
binding.

We will design a language for tactics called \textbf{Nominal LCF} which
supports a distinction between names bound and names taken as parameters, and
then show how it can be elaborated into standard LCF.

\section{The LCF Tactic Language}
\newcommand\MLTypeDecl[1]{\mathtt{type}\ #1}
\newcommand\PartialFun[2]{#1\rightharpoonup #2}
\newcommand\LCFJudgment{\mathit{judgment}}
\newcommand\LCFEvidence{\mathit{evidence}}
\newcommand\LCFTactic{\mathit{tactic}}
\newcommand\MLList[1]{#1\;\mathit{list}}
\newcommand\MLAtom{\mathbb{A}}
\newcommand\MLNat{\mathbb{N}}
\newcommand\MLStream[1]{{#1}^{\MLNat}}
\newcommand\LCFAtomicTactic{\mathit{atactic}}
\newcommand\StreamTake[2]{\bar{#1}(#2)}

The essence of the LCF tactic system is captured in the following (idealized)
ML signature:

\[
  \begin{array}{lcl}
    \MLTypeDecl{\LCFJudgment}\\
    \MLTypeDecl{\LCFEvidence}\\
    \MLTypeDecl{\LCFTactic} &= &\PartialFun{\LCFJudgment}{\MLList{\LCFJudgment} \otimes (\MLList{\LCFEvidence}\to\LCFEvidence)}
  \end{array}
\]

In other words, a tactic is a partial function that takes a goal to its
subgoals, and specifies how to transform the evidence of its subgoals into the
evidence for the main goal. In the case of the sequence calculus we were
considering, \verb!judgment! would be a type of sequents.  Now, in general a
tactic may need to consume names from a \emph{name store}, which is an infinite
stream of \emph{atoms} or \emph{symbols}:

\[
  \begin{array}{lcl}
    \MLTypeDecl{\MLAtom}\\
    \MLTypeDecl{\LCFAtomicTactic} &= &\MLStream{\MLAtom} \to \LCFTactic
  \end{array}
\]

\newcommand\KleeneEq[2]{\IMode{#1}\approx \IMode{#2}}
\newcommand\UMod[2]{\mathsf{umod}(#1,#2)}

\subsection{Atomic tactics and their uniform continuity}

Left sequent rules can be coded as so-called \emph{atomic tactics}, tactics which
consume a stream of names. In fact, every such tactic is \emph{uniformly
continuous} in a specific sense. For an atomic sequence
$\Member{\alpha}{\MLStream{\MLAtom}}$ and a natural number
$\Member{n}{\MLNat}$, let $\StreamTake{\alpha}{n}$ be the initial segment of
$\alpha$ of length $n$. Let $\KleeneEq{M}{N}$ be \emph{observational equivalence}:
$M$ evaluate to the same value, or they diverge.

Then, for any atomic tactic $T$ and judgment $\mathcal{J}$, we can calculate a
uniform modulus of continuity:
\begin{equation}
  \exists \OMode{n}\in\IMode{\MLNat}.
  \forall \Member{\alpha,\beta}{\MLStream{\MLAtom}}.
    \IsEq{\bar{\alpha}(n)}{\bar{\beta}(n)}\implies \KleeneEq{T(\alpha,\mathcal{J})}{T(\beta,\mathcal{J})}
  \tag{uniform continuity}
\end{equation}

This calculation can be realized computationally in our metalanguage in a
number of ways, but for our purposes it suffices to remark that it is a
consequence of Brouwer's Fan Theorem, which we hold to be evident. Let
$\Member{\UMod{T}{\mathcal{J}}}{\MLNat}$ be the uniform modulus of continuity
for a tactic $T$ at goal $\mathcal{J}$.

\section{Nominal LCF}

\subsection{Static Semantics}

\newcommand\SortATac{\mathtt{atac}}
\newcommand\SortBTac{\mathtt{btac}}
\newcommand\SortTac{\mathtt{tac}}
\newcommand\SortMTac{\mathtt{mtac}}
\newcommand\SortHyp{\mathtt{hyp}}

We will define the \textbf{Nominal LCF} language by specifying an abt signature
for it; at its heart is the refactoring of the various sequencing tacticals
$\mathtt{THEN}$, $\mathtt{THENL}$, etc.\ of LCF into a single sequencing
tactical combined a separate notion of \emph{multi-tactic}.

First, we define sorts for tactics, atomic tactics, binding tactics and multi-tactics
respectively:
\begin{gather*}
  \infer{
    \IsSort{\SortTac}
  }{}
  \qquad
  \infer{
    \IsSort{\SortATac}
  }{}
  \qquad
  \infer{
    \IsSort{\SortBTac}
  }{}
  \qquad
  \infer{
    \IsSort{\SortMTac}
  }{}
\end{gather*}

We also provide a sort to classify hypotheses:
\[
  \infer{
    \IsSort{\SortHyp}
  }{
  }
\]

Now, we'll define the operators of \textbf{Nominal LCF}; note that the
operators of sort $\SortATac$ are arbitrary and are provided only for the sake
of illustration. We will consider the definition of \textbf{Nominal LCF} as
over some signature $\Sigma$ of atomic tactics.

\[
  \infer{
    \IsOperator{\Upsilon}{\mathtt{id}}{
      \MkArity{}{\SortATac}
    }
  }{
  }
  \qquad
  \infer{
    \IsOperator{\Upsilon}{\mathtt{fail}}{
      \MkArity{}{\SortATac}
    }
  }{
  }
  \qquad
  \infer{
    \IsOperator{\Upsilon,a:\SortHyp}{\mathtt{elim}[a]}{
      \MkArity{}{\SortATac}
    }
  }{
  }
\]

\[
  \infer{
    \IsOperator{\Upsilon}{\mathtt{seq}_n}{
      \MkArity{
        \MkValence{}{}{\SortBTac},
        \MkValence{\SortHyp^n}{}{\SortMTac}
      }{\SortTac}
    }
  }{
    \Member{n}{\MLNat}
  }
\]

\[
  \infer{
    \IsOperator{\Upsilon}{\mathtt{smash}}{
      \MkArity{
        \MkValence{}{}{\SortATac},
        \MkValence{}{}{\SortATac}
      }{\SortBTac}
    }
  }{
    \Member{n}{\MLNat}
  }
\]

\[
  \infer{
    \IsOperator{\Upsilon}{\mathtt{all}}{
      \MkArity{
        \MkValence{}{}{\SortTac}
      }{\SortMTac}
    }
  }{
  }
  \qquad
  \infer{
    \IsOperator{\Upsilon}{\mathtt{each}_n}{
      \MkArity{
        \MkValence{}{}{\SortTac}^n
      }{\SortMTac}
    }
  }{
    \Member{n}{\MLNat}
  }
  \qquad
  \infer{
    \IsOperator{\Upsilon}{\mathtt{some}_i}{
      \MkArity{
        \MkValence{}{}{\SortTac}
      }{\SortMTac}
    }
  }{
    \Member{i}{\MLNat}
  }
\]

\newcommand\TacSmash[2]{#1\supsetplus #2}
\newcommand\TacEach[1]{\llparenthesis #1\rrparenthesis}
\newcommand\TacSeq[3]{#2\leftarrow #1; #3}

For the sake of clarity, we introduce the following notational abbreviations
for tactic expressions:
\begin{align*}
  \ADefine{
    t_1;t_2
  }{
    \App{\mathtt{seq}_0}{\MkAbs{}{}{t_1};\MkAbs{}{}{t_2}}
  }\\
  \ADefine{
    \TacSeq{t_1}{a_0,\dots,a_n}{t_2}
  }{
  \App{\mathtt{seq}_n}{\MkAbs{}{}{t_1};\MkAbs{a_0,\dots,a_n}{}{t_2}}
  }\\
  \ADefine{
    \TacSmash{t_1}{t_2}
  }{
    \App{\mathtt{smash}}{\MkAbs{}{}{t_1};\MkAbs{}{}{t_n}}
  }\\
  \ADefine{
    \Box t
  }{
    \App{\mathtt{all}}{\MkAbs{}{}{t}}
  }\\
  \ADefine{
    \TacEach{t_1,\dots,t_n}
  }{
    \App{\mathtt{each}_n}{\MkAbs{}{}{t}}
  }\\
  \ADefine{
    \Diamond_i t
  }{
    \App{\mathtt{some}_i}{\MkAbs{}{}{t}}
  }\\
\end{align*}

\subsection{Dynamic Semantics}
\newcommand\LCFElab[5]{%
  \mathcal{M}\models_\IMode{\rho}
  \IMode{#1}\parallel\IMode{#2}\vdash%
  \IMode{#3}\;@\;\IMode{#4}\Longrightarrow\OMode{#5}%
}

\newcommand\LCFElabSt[7]{%
  \mathcal{M}\models_\IMode{\rho}%
  % \IMode{#1}\parallel\IMode{#2}\vdash%
  \IMode{#3}\;@\; \IMode{#4}\xRightarrow{\;\OMode{#7}\;}\OMode{#5}%
  \ll\IMode{#6}
}

\newcommand\LCFElabStVerbose[7]{%
  \mathcal{M}\models_\IMode{\rho}%
  \IMode{#1}\parallel\IMode{#2}\vdash%
  \IMode{#3}\;@\; \IMode{#4}\xRightarrow{\;\OMode{#7}\;}\OMode{#5}%
  \ll\IMode{#6}
}

For a signature of atomic tactics $\Sigma$, we will define a $\Sigma$-model
$\mathcal{M}$ to be an interpretation of each atomic tactic
$\Member{t}{\Sigma}$ into some
$\Member{\SemBrackets{t}_\mathcal{M}}{\LCFAtomicTactic}$; additionally, the
model shall come equipped with free choice sequence of atoms
$\Member{\alpha_\mathcal{M}}{\MLStream{\MLAtom}}$, such that each neighborhood
$\IMode{\vec{u}}\ni\IMode{\alpha_\mathcal{M}}$ contains all distinct atoms.
Then, we can interpret all of \textbf{Nominal LCF} into $\mathcal{M}$, by
defining an elaboration judgment
$\LCFElab{\Upsilon}{\Gamma}{t}{\mathcal{J}}{P}$ with
$\Member{\mathcal{J}}{\LCFJudgment}$ and
$\Member{P}{\MLList{\LCFJudgment}\otimes(\MLList{\LCFEvidence}\to\LCFEvidence)}$,
such that $\Member{\rho(x)}{\LCFAtomicTactic}$ for each
$\Member{x:\SortATac}{\Gamma}$.

First, we define the elaboration uniformly by appealing to an auxiliary judgment:
\[
  \infer{
    \LCFElab{\Upsilon}{\Gamma}{t}{\mathcal{J}}{P}
  }{
    \LCFElabStVerbose{\Upsilon}{\Gamma}{t}{\mathcal{J}}{P}{\alpha_\mathcal{M}}{\mu}
  }
\]

This auxiliary form of judgment is made with respect to a free choice sequence
of names $\alpha$, and synthesizes the uniform modulus of continuity $\mu$ of
the tactic under consideration.  When it leads to no ambiguity, we will write
$\LCFElabSt{\Upsilon}{\Gamma}{t}{\mathcal{J}}{P}{\alpha}{\mu}$ instead of the
more verbose
$\LCFElabStVerbose{\Upsilon}{\Gamma}{t}{\mathcal{J}}{P}{\alpha}{\mu}$; we will
explain this judgment for $t$ of sorts $\SortATac$, $\SortBTac$, and
$\SortTac$. To start with, we will give the elaboration rules for constants
(atomic tactics) and variables denoting atomic tactics:

\[
  \infer{
    \LCFElabSt{\Upsilon}{\Gamma}{t}{\mathcal{J}}{
      \SemBrackets{t}_\mathcal{M}(\alpha,\mathcal{J})
    }{\alpha}{\mu}
  }{
    \Match{
      \UMod{\SemBrackets{t}_\mathcal{M}}{\mathcal{J}}
    }{
      \mu
    }
  }
  \qquad
  \infer{
    \LCFElabSt{\Upsilon}{\Gamma}{x}{\mathcal{J}}{
      \rho(x)(\alpha,\mathcal{J})
    }{\alpha}{\mu}
  }{
    \Match{
      \UMod{\rho(x)}{\mathcal{J}}
    }{
      \mu
    }
  }
\]

The rule for the \emph{smash} tactical $\TacSmash{t_1}{t_2}$ exploits the
uniform continuity of atomic tactics to divide up the name store $\alpha$
between $t_1$ and $t_2$:
\[
  \infer{
    \LCFElabSt{\Upsilon}{\Gamma}{\TacSmash{t_1}{t_2}}{\mathcal{J}}{
      \Pair{
        \bigoplus_{i\leq n}\vec{\mathcal{K}}_i
      }{
        \lambda(\vec{x}_0,\dots,\vec{x}_n). E[F_0[\vec{x}_0], \dots, F_0[\vec{x}_n]]
      }
    }{\alpha}{\mu'}
  }{
    \begin{array}{l}
      \LCFElabSt{\Upsilon}{\Gamma}{t_1}{\mathcal{J}}{
        \Pair{[\mathcal{J}_0,\dots,\mathcal{J}_n]}{E}
      }{\alpha}{\mu}\\
      \HypJ{
        \LCFElabSt{\Upsilon}{\Gamma}{t_2}{\mathcal{J}_i}{
          \Pair{\vec{\mathcal{K}}_i}{F_i}
        }{\beta}{\mu_i}
      }{
        \Leq{i}{n}
      }
    \end{array} &
    \begin{array}{l}
      \Match{\alpha}{
        [u_0,\dots,u_\mu]\oplus\beta
      }\\
      \Match{
        \mu + \bigsqcup{i\leq n}\mu_i
      }{\mu'}
    \end{array}
  }
\]


Next, we define the sequencing tactical, once for each multi-tactical.

\[
  \infer{
    \LCFElabSt{\Upsilon}{\Gamma}{
      \TacSeq{t_1}{\vec{u}}{\Box t_2}
    }{\mathcal{J}}{
      \Pair{
        \bigoplus_{i\leq n}\vec{\mathcal{K}}_i
      }{
        \lambda(\vec{x}_0,\dots,\vec{x}_n). E[F_0[\vec{x}_0], \dots, F_0[\vec{x}_n]]
      }
    }{\alpha}{\mu'}
  }{
    \begin{array}{l}
      \LCFElabSt{\Upsilon}{\Gamma}{t_1}{\mathcal{J}}{
        \Pair{[\mathcal{J}_0,\dots,\mathcal{J}_n]}{E}
      }{\vec{u}\oplus\alpha}{\mu}\\
      \HypJ{
        \LCFElabSt{\Upsilon,\vec{u}:\SortHyp}{\Gamma}{t_2}{\mathcal{J}_i}{
          \Pair{\vec{\mathcal{K}}_i}{F_i}
        }{\beta}{\mu_i}
      }{
        \Leq{i}{n}
      }
    \end{array} &
    \begin{array}{l}
      \Match{\vec{u}\oplus\alpha}{
        [v_0,\dots,v_\mu]\oplus\beta
      }\\
      \Match{
        \mu + \bigsqcup_{i\leq n}\mu_i
      }{\mu'}
    \end{array}
  }
\]

\[
  \infer{
    \LCFElabSt{\Upsilon}{\Gamma}{
      \TacSeq{t}{\vec{u}}{\TacEach{t_0,\dots,t_n}}
    }{\mathcal{J}}{
      \Pair{
        \bigoplus_{i\leq n}\vec{\mathcal{K}}_i
      }{
        \lambda(\vec{x}_0,\dots,\vec{x}_n). E[F_0[\vec{x}_0], \dots, F_0[\vec{x}_n]]
      }
    }{\alpha}{\mu'}
  }{
    \begin{array}{l}
      \LCFElabSt{\Upsilon}{\Gamma}{t_1}{\mathcal{J}}{
        \Pair{[\mathcal{J}_0,\dots,\mathcal{J}_n]}{E}
      }{\vec{u}\oplus\alpha}{\mu}\\
      \HypJ{
        \LCFElabSt{\Upsilon,\vec{u}:\SortHyp}{\Gamma}{t_i}{\mathcal{J}_i}{
          \Pair{\vec{\mathcal{K}}_i}{F_i}
        }{\beta}{\mu_i}
      }{
        \Leq{i}{n}
      }
    \end{array} &
    \begin{array}{l}
      \Match{\vec{u}\oplus\alpha}{
        [v_0,\dots,v_\mu]\oplus\beta
      }\\
      \Match{
        \mu + \bigsqcup_{i\leq n}\mu_i
      }{\mu'}
    \end{array}
  }
\]

\[
  \infer{
    \LCFElabSt{\Upsilon}{\Gamma}{
      \TacSeq{t_1}{\vec{u}}{\Diamond_i t_2}
    }{\mathcal{J}}{
      \Pair{
        \vec{\mathcal{J}}_<
        \oplus
        \vec{\mathcal{K}}
        \oplus
        \vec{\mathcal{J}}_>
      }{
        \lambda\vec{x}.E(\vec{x}_<\oplus[F]\oplus\vec{x}_>)
      }
    }{\alpha}{\mu'}
  }{
    \begin{array}{l}
      \LCFElabSt{\Upsilon}{\Gamma}{t_1}{\mathcal{J}}{
        \Pair{\vec{\mathcal{J}}}{E}
      }{\vec{u}\oplus\alpha}{\mu_1}\\
      \LCFElabSt{\Upsilon,\vec{u}:\SortHyp}{\Gamma}{t_2}{\mathcal{J}_i}{
        \Pair{\vec{\mathcal{K}}}{F}
      }{\beta}{\mu_2}
    \end{array} &
    \begin{array}{l}
      \Match{\vec{u}\oplus\alpha}{
        [v_0,\dots,v_{\mu_1}]\oplus\beta
      }\\
      \Match{\mu_1+\mu_2}{\mu'}
    \end{array} &
    \begin{array}{l}
      \Match{
        \vec{\mathcal{J}}
      }{
        \vec{\mathcal{J}}_<\oplus[\vec{\mathcal{J}}_i]\oplus\vec{\mathcal{J}}_>
      }\\
      \Match{
        \vec{x}
      }{
        \vec{x}_<\oplus[\vec{x}_i]\oplus\vec{x}_>
      }
    \end{array}
  }
\]

