\chapter{The Refinement Logic}

\newcommand\PRLSequent[6]{
  \IMode{#1}\mid%
  \IMode{#2}\gg_{\IMode{#3}}^{\IMode{#4}}%
  \IMode{#5}\leadsto%
  \OMode{#6}%
}

\newcommand\IsHypCtx[3]{
  \IMode{#1}\triangleright%
  \IMode{#2}\parallel%
  \IMode{#3}\ \mathit{hypctx}%
}

\section{Forms of Judgment}

\subsection{Hypothesis Contexts}
First, we develop the syntax of hypothesis contexts with respect to a
metavariable context $\IsMetaCtx{\Theta}$ and a symbol context
$\IsSymCtx{\Upsilon}$:

\[
  \infer{
    \IsHypCtx{\Theta}{\Upsilon}{\cdot}
  }{}
  \qquad
  \infer{
    \IsHypCtx{\Theta}{\Upsilon}{H, x:_\tau A}
  }{
    \begin{array}{l}
      \IsHypCtx{\Theta}{\Upsilon}{H}\\
      \IsAbt{\Theta}{\Upsilon}{\Dom{H}}{A}{\AbtSortExp}
    \end{array}&
    \NotIn{x}{H}
  }
\]

The \emph{domain} $\IsVarCtx{\Dom{H}}$ of a hypothesis context
$\IsHypCtx{\Theta}{\Upsilon}{H}$ is defined as follows:
\begin{align*}
  \ADefine{\Dom{\cdot}}{\cdot}\\
  \ADefine{\Dom{H,x:_\tau A}}{\Dom{H}, x : \tau}
\end{align*}

\subsection{Hypothetico-General Sequents}

\RedPRL's refinement logic is organized around several forms of \emph{sequent},
which are schematized as follows, depending on the form of conclusion:
\[
  \PRLSequent{\Gamma}{H}{\Theta}{\Upsilon}{\textit{conclusion}}{\textit{synthesis}}
\]

\newcommand\IsConcl[5]{%
  \IMode{#1}\triangleright%
  \IMode{#2}\parallel%
  \IMode{#3}\vdash%
  \lfloor\IMode{#4}\rfloor\ \mathit{concl}%
  \leadsto \OMode{#5}%
}

\subsubsection{Forms of Conclusion}
The form of synthesis depends on the particular form of conclusion (categorical
judgment); we formalize this with the
$\IsConcl{\Theta}{\Upsilon}{\Gamma}{C}{\tau}$ judgment form, defined as
follows:

\[
  \infer[\text{Truth}]{
    \IsConcl{\Theta}{\Upsilon}{\Gamma}{A\ \textit{true}_\tau}{\tau}
  }{
    \IsAbt{\Theta}{\Upsilon}{\Gamma}{A}{\AbtSortExp}
  }
\]


\[
  \infer[\text{Equality}]{
    \IsConcl{\Theta}{\Upsilon}{\Gamma}{M = N \in_\tau A}{\AbtSortExp}
  }{
    \begin{array}{l}
      \IsAbt{\Theta}{\Upsilon}{\Gamma}{M}{\tau}\\
      \IsAbt{\Theta}{\Upsilon}{\Gamma}{N}{\tau}
    \end{array} &
    \IsAbt{\Theta}{\Upsilon}{\Gamma}{A}{\AbtSortExp}
  }
\]

\[
  \infer[\text{Membership}]{
    \IsConcl{\Theta}{\Upsilon}{\Gamma}{M \in_\tau A}{\AbtSortExp}
  }{
    \IsAbt{\Theta}{\Upsilon}{\Gamma}{M}{\tau} &
    \IsAbt{\Theta}{\Upsilon}{\Gamma}{A}{\AbtSortExp}
  }
\]

\[
  \infer[\text{Membership Synthesis}]{
    \IsConcl{\Theta}{\Upsilon}{\Gamma}{R\ \textit{synth}_\tau}{\AbtSortExp}
  }{
    \IsAbt{\Theta}{\Upsilon}{\Gamma}{R}{\tau} &
    \IsAbt{\Theta}{\Upsilon}{\Gamma}{A}{\AbtSortExp}
  }
\]

\[
  \infer[\text{Equality Synthesis}]{
    \IsConcl{\Theta}{\Upsilon}{\Gamma}{R = S\ \textit{synth}_\tau}{\AbtSortExp}
  }{
    \begin{array}{l}
      \IsAbt{\Theta}{\Upsilon}{\Gamma}{R}{\tau}\\
      \IsAbt{\Theta}{\Upsilon}{\Gamma}{S}{\tau}
    \end{array} &
    \IsAbt{\Theta}{\Upsilon}{\Gamma}{A}{\AbtSortExp}
  }
\]

\[
  \infer[\text{Level Synthesis}]{
    \IsConcl{\Theta}{\Upsilon}{\Gamma}{A\ \textit{type}}{\AbtSortLvl}
  }{
    \IsAbt{\Theta}{\Upsilon}{\Gamma}{A}{\AbtSortExp}
  }
\]

\subsubsection{Syntax of Sequents}

Now that we have enumerated the forms of conclusion, we can precisely give the
syntax of the sequent judgment. We will say that the sequent
$\PRLSequent{\Gamma}{H}{\Theta}{\Upsilon}{C}{S}$ is a meaningful judgment in
case the following presuppositions obtain:
\begin{enumerate}
  \item $\IsMetaCtx{\Theta}$
  \item $\IsSymCtx{\Upsilon}$
  \item $\IsHypCtx{\Theta}{\Upsilon}{H}$
  \item $\IsVarCtx{\Gamma}$ and $\IMode{\Gamma}\subseteq\IMode{\Dom{H}}$
  \item $\IsConcl{\Theta}{\Upsilon}{\Dom{H}}{C}{\tau}$
  \item $\IsAbs{\Theta}{\Upsilon}{\Dom{H}\setminus \Gamma}{S}{\MkValence{}{\Gamma}{\tau}}$
\end{enumerate}

The meaning of the sequent judgment is given inductively by a collection of
formal rules, which are then related to the \emph{intended semantics} of
Nominal Computational Type Theory by a soundness theorem.

\section{Selected Rules}

\[
  \infer{
    \PRLSequent{\Gamma}{H}{\Theta}{\Upsilon}{M \in_\tau A}{\MkAbs{}{\vec{x}}{\mathtt{Ax}}}
  }{
    \PRLSequent{\Gamma}{H}{\Theta}{\Upsilon}{M = M \in_\tau A}{\MkAbs{}{\vec{x}}{\mathtt{Ax}}}
  }
\]

\[
  \infer{
    \PRLSequent{\Gamma}{H}{\Theta}{\Upsilon}{R = S\ \textit{synth}_\tau}{\MkAbs{}{\vec{x}}{A_1}}
  }{
    \begin{array}{l}
      \PRLSequent{\Gamma}{H}{\Theta}{\Upsilon}{R\ \textit{synth}_\tau}{\MkAbs{}{\vec{x}}{A_1}}\\
      \PRLSequent{\Gamma}{H}{\Theta}{\Upsilon}{S\ \textit{synth}_\tau}{\MkAbs{}{\vec{x}}{A_2}}
    \end{array}
    \begin{array}{l}
      \PRLSequent{\Gamma}{H}{\Theta}{\Upsilon}{A_1\ \textit{type}}{\MkAbs{}{\vec{x}}{i}}\\
      \PRLSequent{\Gamma}{H}{\Theta}{\Upsilon}{A_1=A_2\in_\AbtSortExp \mathbb{U}_i}{\MkAbs{}{\vec{x}}{\mathtt{Ax}}}
    \end{array}
  }
\]

